\documentclass{main}

\title{数学物理笔记}

\begin{document}

\begin{titlepage}
    \centering

    \vspace*{5cm}
    {\Huge \bfseries 数学物理笔记}\\[1cm] 
    
    {\LARGE 数学物理基础与经典数值方法总结}\\[2cm]
\end{titlepage}

\tableofcontents
\newpage

%--------------------------------------------------------------
% 逻辑学部分
%--------------------------------------------------------------

% 逻辑学简介
\part{逻辑学部分}

\chapter{逻辑学简介}

%--------------------------------------------------------------
% 数学部分
%--------------------------------------------------------------

% 集合论
\chapter{集合论}

\section{集合公理}

% 结构与范畴
\chapter{结构与范畴}

\section{数学结构}

\subsection{二元关系}

\subsection{代数结构}

\subsection{序结构}

\subsection{拓扑结构}

% \subsection{其他结构}

% \begin{definition}[代数 Algebra]
    
% \end{definition}

% \begin{definition}[李群 Lie Group]
    
% \end{definition}

\newpage
\section{范畴论}

\newpage

% 数系扩充
\chapter{数系扩充}

自然数系是最基础的数系,通过集合论,已经定义了自然数。自然数上定义加法和乘法两种运算,通过这两种运算的组合,可以定义多项式和多项式方程。

\begin{definition}[多项式 Polynomial]
    设 $a_0,a_1,a_2,\cdots,a_n$ 是常量,$x$ 是变量,$n\in\mathbb{N}$ 是自然数,则形如
    \[
        P(x) = a_n x^n + a_{n-1} x^{n-1} + \cdots + a_1 x + a_0
    \]
    的表达式称为一个\textbf{多项式},其中,$a_i,i=0,\cdots,n$ 称为多项式的\textbf{系数},$n$ 称为多项式的\textbf{次数},当 $a_n \neq 0$ 时,称 $a_n$ 为多项式的\textbf{首项系数}。
    \label{def:polynomial_real}
\end{definition}

\begin{definition}[多项式方程 Polynomial Equation]
    设 $P(x)$ 是一个首项系数不为零的多项式,等式:
    \begin{equation}
        P(x) = a_n x^n + a_{n-1} x^{n-1} + \cdots + a_1 x + a_0 = 0
        \label{eq:polynomial_equation}
    \end{equation}
    称为一个 $n$-\textbf{次多项式方程}。
    \label{def:polynomial_equation_real}
\end{definition}

数系扩充的动力源自解多项式方程 \ref{eq:polynomial_equation},从自然数出发可以逐步扩充出整数、有理数、实数和复数。根据代数基本定理,复数是多项式方程的代数闭包,也即任何系数为复数的多项式方程都能在复数域内找到解。从代数角度,数系扩充到复数就足够了。
\vspace{1em}

\begin{note}
    在复数之后,数系的扩充主要有两条途径:一是引入超实数和超复数,从而构造非标准分析;二是引入四元数和八元数,从而构造更高维的代数结构。不过这些数系都与解多项式方程的关系不大,最后只简单介绍一下四元数系。
\end{note}

\newpage

\section{整数}

为了解多项式方程:$x + a = 0,\ \forall a \in \mathbb{N}$ 需要引入整数系。

\newpage
\section{有理数}

为了解多项式方程:$ax + b = 0,\ \forall a,b \in \mathbb{Z}, a \neq 0$ 需要引入有理数系。

\newpage


\section{实数}

为了解多项式方程:$x^2 - 2 = 0$ 需要引入实数系。实数的定义没有那么简单,在历史上曾多次引发争议。现代比较公认的实数构造方法主要有两种:戴德金分割法和柯西列法。这里介绍柯西列法,通过下面这个例子来简单理解这种方法的思想。

\begin{example}
    函数 $f(x)=x^2-2$ 的零点对应多项式方程 $x^2 - 2 = 0$ 的根,通过牛顿迭代法,可以找到一个有理数序列 $a_n$,使得 $(a_n)^2$ 趋近于 $2$:
    \begin{equation}
        a_{n+1} = \frac{1}{2}\left(a_n + \frac{2}{a_n}\right)
        \label{eq:iteration_method_for_square_root_of_2}
    \end{equation}
\end{example}




\newpage
\section{复数}

为了解多项式方程 $x^2 + 1 = 0$ 需要引入复数系。

\subsection{代数基本定理}

\newpage
\section{四元数}

\newpage

% 抽象代数
\chapter{抽象代数}

抽象代数起源于多项式方程根式解的研究。

% 线性代数
\chapter{线性代数}

线性代数是现代数学物理最基础的“语言”,其重要性不言而喻。

%--------------------------------------------------------------
% 物理学部分
%--------------------------------------------------------------

% 质点力学
\part{物理学部分}

\chapter{质点力学}

\section{质点运动学}

% 分析力学
\chapter{分析力学}

% 电动力学

% 光学

%--------------------------------------------------------------
% 数值方法部分
%--------------------------------------------------------------

% 浮点数与误差分析
\part{数值方法部分}

\chapter{浮点数与误差分析}

\section{浮点数表示}

% 函数的近似

\end{document}