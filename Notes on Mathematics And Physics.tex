\documentclass{main}

\title{数学物理笔记}

\begin{document}

\begin{titlepage}
    \centering

    \vspace*{5cm} % 顶部留白(*表示忽略段落间距限制)
    {\Huge \bfseries 数学物理笔记}\\[1cm] % \bfseries 加粗,[1cm] 标题下方间距
    
    {\LARGE 数学物理基础与经典数值方法总结}\\[2cm] % \LARGE 比 \Huge 小一级
\end{titlepage}

\tableofcontents
\newpage

%--------------------------------------------------------------
% 逻辑学部分
%--------------------------------------------------------------

% 逻辑学简介
\part{逻辑学部分}

\chapter{逻辑学简介}

%--------------------------------------------------------------
% 数学部分
%--------------------------------------------------------------

% 集合与关系
\part{数学部分}

\chapter{集合与关系}

\section{集合公理}

% 数系扩充
\chapter{数系扩充}

自然数系是最基础的数系,通过集合论,已经定义了自然数。自然数上定义加法和乘法两种运算,通过这两种运算的组合,可以定义多项式和多项式方程。

\begin{definition}[多项式 Polynomial]
    设 $a_0,a_1,a_2,\cdots,a_n$ 是常量,$x$ 是变量,$n\in\mathbb{N}$ 是自然数,则形如
    \[
        P(x) = a_n x^n + a_{n-1} x^{n-1} + \cdots + a_1 x + a_0
    \]
    的表达式称为一个\textbf{多项式},其中,$a_i,i=0,\cdots,n$ 称为多项式的\textbf{系数},$n$ 称为多项式的\textbf{次数},当 $a_n \neq 0$ 时,称 $a_n$ 为多项式的\textbf{首项系数}。
    \label{def:polynomial_real}
\end{definition}

\begin{definition}[多项式方程 Polynomial Equation]
    设 $P(x)$ 是一个首项系数不为零的多项式,等式:
    \begin{equation}
        P(x) = a_n x^n + a_{n-1} x^{n-1} + \cdots + a_1 x + a_0 = 0
        \label{eq:polynomial_equation}
    \end{equation}
    称为一个 $n$-\textbf{次多项式方程}。
    \label{def:polynomial_equation_real}
\end{definition}

数系扩充的动力源自解多项式方程 \ref{eq:polynomial_equation},从自然数出发可以逐步扩充出整数、有理数、实数和复数。根据代数基本定理,复数是多项式方程的代数闭包,也即任何系数为复数的多项式方程都能在复数域内找到解。从代数角度,数系扩充到复数就足够了。
\vspace{1em}

\begin{note}
    在复数之后,数系的扩充主要有两条途径:一是引入超实数和超复数,从而构造非标准分析;二是引入四元数和八元数,从而构造更高维的代数结构。不过这些数系都与解多项式方程的关系不大,最后只简单介绍一下四元数系。
\end{note}

\newpage

\section{整数}

为了解多项式方程:$x + a = 0,\ \forall a \in \mathbb{N}$ 需要引入整数系。

\newpage

%--------------------------------------------------------------
% 物理学部分
%--------------------------------------------------------------

% 质点力学
\part{物理学部分}

\chapter{质点力学}

\section{质点运动学}

% 分析力学
\chapter{分析力学}

% 电动力学

% 光学

%--------------------------------------------------------------
% 数值方法部分
%--------------------------------------------------------------

% 浮点数与误差分析
\part{数值方法部分}

\chapter{浮点数与误差分析}

\section{浮点数表示}

% 函数的近似

\end{document}